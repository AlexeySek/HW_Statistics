%I used the template from Coursera course on LaTex by Danil Fyodorovykh%

\documentclass[a4paper]{article} % this is used for comments
\usepackage[utf8]{inputenc}
%%% Дополнительная работа с математикой
\usepackage{amsmath,amsfonts,amssymb,amsthm,mathtools} % AMS
\usepackage{icomma} % "Умная" запятая: $0,2$ --- число, $0, 2$ --- перечисление
\usepackage[english,russian]{babel}

%% Номера формул
\mathtoolsset{showonlyrefs=true} % Показывать номера только у тех формул, на которые есть \eqref{} в тексте.

%% Шрифты
\usepackage{euscript}	 % Шрифт Евклид
\usepackage{mathrsfs} % Красивый матшрифт

%% Свои команды
\DeclareMathOperator{\sgn}{\mathop{sgn}}

%% Перенос знаков в формулах (по Львовскому)
\newcommand*{\hm}[1]{#1\nobreak\discretionary{}
{\hbox{$\mathsurround=0pt #1$}}{}}




\title{Промежуточный экзамен 2016-2017}
\author{Алексей Сек, БЭК182}
\date{\today}






\begin{document}

\maketitle

\textbf{Промежуточный экзамен 2016-2017}

\begin{enumerate}
    \item
    Известно, что некоторые события $A$ и $B$ независимы, если выполняется такое условие:
    $ P(A \cap B) = P(A) \cdot P(B) $, верно и обратное: если данное условие не выполняется - события являются зависимыми. Исходя из этого, решим задачу:
    
    \textbf{Посчитаем вероятности каждого из событий отдельно:}
    
    Вполне очевидно, что троек в колоде ровно 4, тогда по классическому определению вероятности $ P(A) = \frac{4}{52} = \frac{1}{13} $
    
    Семерок в колоде, что также очевидно, ровно 4, но т.к. событие заключается в вытаскивании второй карты, то вероятность будет иной.
    Примем во внимание, что первой картой могла быть как семерка, так и не семерка, вычислим вероятность события $ B$: $P(B) = \frac{4}{52}\cdot\frac{3}{51} + \frac{48}{52}\cdot\frac{4}{51} $.
    
    Если третья карта - дама пик, то первые две карты - не дамы пик, а дама пик в колоде всего одна, следовательно: $ P(C) = \frac{51}{52}\cdot\frac{50}{51}\cdot\frac{1}{50} $
    
    \textbf{Посчитаем все произведения пар событий (от перестановок множителей сумма не меняется - поэтому считаем только 3 пары):}
    
    $ P(A) \cdot P(B) = \frac{4}{52}\cdot(\frac{4}{52}\cdot\frac{3}{51} + \frac{48}{52}\cdot\frac{4}{51}) $
    
    $ P(A) \cdot P(C) = \frac{4}{52}\cdot\frac{51}{52}\cdot\frac{50}{51}\cdot\frac{1}{50} $
    
    $ P(B) \cdot P(C) = (\frac{4}{52}\cdot\frac{3}{51} + \frac{48}{52}\cdot\frac{4}{51})\cdot\frac{51}{52}\cdot\frac{50}{51}\cdot\frac{1}{50} $
    
    \textbf{Теперь посчитаем пересечения рассмотренных событий:}
    
    $ P(A \cap B) = \frac{4}{52}\cdot\frac{4}{51}$ (первая карта - тройка, вторая - семерка)
    
    $ P(A \cap C) = \frac{4}{52}\cdot\frac{50}{51}\cdot\frac{1}{50} $ (первая карта - тройка, третья - дама пик)
    
    $P(B \cap C) = \frac{4}{52}\cdot\frac{3}{51}\cdot\frac{1}{50} + \frac{47}{52}\cdot\frac{4}{51}\cdot\frac{1}{50}$ (вторая - семерка, третья - дама пик. Во второй части мы считаем, что первая карта не семерка и не дама пик, таких карт $52-4-1=47$)
    
    \textbf{Сравним вероятности пересечений событий и произведения вероятностей этих событий:}
    
    $ P(A) \cdot P(B) \neq P(A \cap B) $ $\Rightarrow$ $A$ и $B$ - зависимые события
    
    $ P(A) \cdot P(C) \neq P(A \cap C) $ $\Rightarrow$ $A$ и $C$ - зависимые события
    
    $ P(B) \cdot P(C) \neq P(B \cap C) $ $\Rightarrow$ $B$ и $C$ - зависимые события
    
    \textbf{Ответ:} B. События $A$ и $B$ зависимы, события $B$ и $C$ зависимы
    
    
    
    
    \item
    Известно, что функция плотности $f(x)$ обладает следующими свойствами:
    \begin{itemize}
        \item $f(x) \ge 0$, для $ \forall x$ - вероятность не может быть отрицательной
        \item $f(x)$ непрерывна в области опрелеления
        \item $ \int_{-\infty}^{+\infty} f(x) dx = 1$ - условие нормировки (вероятность от 0 до 1)
    \end{itemize}
    
    \textbf{Рассмотрим каждую из функций на выполнение указанных свойств:}
    
    A. $f(x) = -1 \le 0$ $\Rightarrow$ не подходит
    
    B. $f(x) \le 0$ например, при $x = 0$ $\Rightarrow$ не подходит
    
    C. $ \int_{-\infty}^{+\infty} f(x) dx = \int_{-1}^{+\infty} dx/x^2 = 1 $ $\Rightarrow$ все свойтсва соблюдаются
    
    D. Данная функция похожа на функцию плотности для нормального распределения, но из-за отсутствия делителя в степени экспоненты - интеграл нельзя посчитать $\Rightarrow$ не подходит
    
    E. $ \int_{-\infty}^{+\infty} f(x) dx = \int_{0}^{2} x^2 dx = \frac{8}{3} \neq 1 $ $\Rightarrow$ не подходит
    
    \textbf{Ответ:} C.
    
    
    
    \item
    Известно, что $\mathbb{E}(XY) = COV(X,Y) + \mathbb{E}(X)\mathbb{E}(Y)$
    
    Откуда $\mathbb{E}(XY) = 2 + 3\cdot2 = 8$
    
    \textbf{Ответ:} A.
    
    
    
    
    \item 
    Известно, что $corr(X,Y) = \frac{cov(X,Y)}{\sqrt{Var(X)}\sqrt{Var(Y}}$
    
    Откуда $corr(X,Y) = \frac{2}{\sqrt{12}\sqrt{1}} = \frac{1}{\sqrt{3}}$
    
    \textbf{Ответ:} A.
    
    
    
    
    
    \item
    Изввестно, что $Var(aX + bY + c) = a^2 Var(X) + b^2 Var(Y) + 2abCov(X,Y)$, где $a, b, c = const, X, Y$ - некоторые случайные величины
    
    Откуда $Var(2X - Y + 4) = 2^2 \cdot 12 + 1 \cdot 1 - 4 \cdot 2 = 48 + 1 - 8 = 41$
    
    \textbf{Ответ:} E.
    
    
    
    \item
    Известно, что в ковариационной матрице на главной диагонали стоят дисперсии случайных величин, а на побочной - ковариации данных случайных величин друг с другом
    
    Если матрица единичная, то на главной диагонали стоят единицы: $Var(X) = 1$ и $Var(Y) = 1$, а на побочной - нули:  $Cov(X,Y) = 0$
    
    Если ковариация равна нулю, то случайные величины независимы, что и утверждается в варианте D
    
    \textbf{Ответ:} D.
    
    
    
    
    
    \item
    Для решения вспомним свойства корреляции и ковариации:
    
    $Corr(X+Y, 2Y-7) = \frac{Cov(X+Y, 2Y-7)}{\sqrt{Var(X+Y)}\sqrt{Var(2Y-7)}} = \frac{\frac{Cov(X, 2Y) + 2 Var(Y)}{\sqrt{Var(X)}\sqrt{Var(Y)}}}{\frac{\sqrt{Var(X) + Var(Y) + 2Cov(X,Y)}\sqrt{4Var(Y)}}{\sqrt{Var(X)}\sqrt{Var(Y)}}} = \frac{\frac{2 Cov(X, Y)}{Var(X) + Var(Y)} + \frac{2 Var(Y)}{Var(X) + Var(Y)}}{2\sqrt{1 + \frac{Var(Y)}{Var(X)} + \frac{2Cov(X,Y)}{Var(X)}}} = \frac{2 \cdot 0.5 + 2 \cdot \sqrt{\frac{Var(Y)}{Var(Y)}}}{2\sqrt{1 + 1 + 1}} = \frac{1 + 2}{2 \sqrt{3}} = \frac{\sqrt{3}}{2}$
    
    
    Промежуточные подсчеты: $Corr(X,Y) = 0.5, Var(X) = Var(Y)$ $\Rightarrow$ $Cov(X,Y) = 0.5\sqrt{Var(X)}\sqrt{Var(Y)}$ 
    
    \textbf{Ответ:} B.
    
    
    
    \item
    И так несложная задача сильно упроститься, если представить отрезок от 0 до 1, на котором случайная величина $\xi$ может равномерно распределена. С какой же вероятностью она попадет в часть этого отрезка, ограниченную точками $0.2$ и $0.7$, очевидно, что длина такой части отрезка равна $0.5$, когда длина всего отрезка от $0$ и $1$ равна 1
    
    Очевидно: $P(0.2 < \xi < 0.7) = \frac{0.7 - 0.2}{1 - 0} = 0.5 $
    
    \textbf{Ответ:} D.
    
    \item
    По ЦПТ указанное распределение сходится к стандартному нормальному распределению $N(0,1)$ при $n \rightarrow \infty$
    
    Нам известна функция плотности стандартной нормальной случайной величины, а вероятность, как известно, находится как определенный интеграл от функции плотности. 
    
    Пределы интеграрирования равны: $1$ и ${+\infty}$, т.к. исходя из условия нам интересны значения $x >1$
    
    Очевидно, что подойти может только ответ B, который представляет интеграл с правильными пределами интегрирования от правильной функцмм плотности
    
    \textbf{Ответ:} B.
    
    
    
    
    \item
    \textbf{Известно из условия, что:}
    
    $\mathbb{E}(S_n) = 400$
    
    $Var(S_n) = Var(\sum X_i) = n * Var(X_i) = 40000$ т.к. с.в. независимы
    
    $n = 100$
    
    \textbf{Применим ЦПТ:}
    
    $P(S_n > 40400) = P(\frac{S_n - \mathbb{E}(S_n)}{\sqrt{Var(S_n)}} > \frac{40400 - 400}{\sqrt{40000}}) = P(N(0,1) > 2) = 1 - P(N(0,1) < 2) = 1 - 0.9772 = 0.0228$
    
    В ответах указать приближенный ответ, там есть 0.0227, что нам подходит
    
    \textbf{Ответ:} D.
    
    
    
    
    
    \item
    По условию дано:
    
    $\mathbb{E}(X) = 10000$
    
    \textbf{Неравенство Макркова:}
    
    $P(\mid X \mid > a) \leq \frac{\mathbb{E}(X)}{a}$
    
    \textbf{В нашем случае:} 
    
    $P(\mid X \mid > 50000) \leq \frac{10000}{50000} = 0.2$
    
    \textbf{Ответ:} A.
    
    
    
    
    \item
    Несовместность $\Leftrightarrow$ события не могут произойти вместе
    
    \textbf{Всего 3 подбрасывания монеты:}
    
    Событие A: хотя бы 1 раз решка

    Событие B: хотя бы 1 раз орел

    Событие C: все три раза орел
    
    \textbf{Применим здравый смысл:}
    
    A и B одновременно произойти могут: например, 1 орел, 2 решки $\Rightarrow$ Совместны
    
    A и C одновременно произойти не могут: если 3 орла из 3, то решек 0 $\Rightarrow$ Несовместны
    
    B и C одноврменно произойти могут: если 3 раза орел, то хотя бы 1 орел точно есть $\Rightarrow$ Совместны
    
    
    
    
    \textbf{Ответ:} A.
    
    
    
    \item
    По условию дано:
    
    $\mathbb{E}(X) = 50000$
    
    $Var(X) = 10000^2 = \sigma^2 $ Дано стандартное отклонение, а нужна дисперсия
    
    \textbf{Неравенство Чебышёва:} 
    
    $P(\mid X - \mathbb{E}(X) \mid > a) \leq \frac{Var(X)}{a^2}$
    
    \textbf{В нашем случае:} 
    
    $P(\mid X - 50000 \mid \leq 20000) = 1 -  P(\mid X - 50000 \mid > 20000) = 1 -  \frac{10^8}{(2* 10^4)^2} = 1 - 0.25 = 0.75 $
    
    
    \textbf{Ответ:} C.
    
    
    \item
    Мат. ожидание биномиального распределения: $\mathbb{E}((\xi_1)^{2016}) = 0.6$
    
    \textbf{По Закону Больших Чисел:}
    
    $ p\lim_{n\to\infty} \frac{(\xi_1)^{2016} + ... + (\xi_n)^{2016}}{n} = \mathbb{E}((\xi_1)^{2016}) = 0.6 $ 
    
    
    frac{1}{6}
    
    
    
    
    
    
    
    
    \item
    \textbf{Кубик подбросили $5$ раз:}
    
    Ровно $2$ раза шестерка $\Rightarrow$ $2$ раза шестерка и  $3$ раза не шестерка
     
    События независимые $\Rightarrow$ можем взять их произведение
    
    Пусть $X$ количество выпавших шестерок, тогда:
    
    \textbf{Биномиальное распределение, поэтому:}
    
    $P(X=2) = C_5^2 \cdot(\frac{1}{6})^2 \cdot (\frac{5}{6})^3 = \frac{5^4}{2^4\cdot3^5}$
    
    \textbf{Ответ:} Нет верного ответа.
    
    
    
    
    
    \item
    \textbf{Кубик подбросили $5$ раз:}
    
    Несложно найти мат. ожидание для одного броска:
    
    $\mathbb{E}(X) = \frac{1}{6}\cdot1 + \frac{5}{6}\cdot0 = \frac{1}{6}$
    
    Посчиатем мат. ожидание 5 бросоков:

    $Y = \sum X_i$ $\Rightarrow$ $\mathbb{E}(Y) = 5\mathbb{E}(X) = \frac{5}{6}$
    
    \textbf{Биномиальное распределение, поэтому:}
    
    $Var(Y) = npq = 5 \cdot \frac{1}{6} \cdot \frac{1}{6} = \frac{25}{36} $ 
    
    \textbf{Ответ:} Нет верного ответа.
    
    
    
    \item
    \textbf{Биномиальное распределение, поэтому:}
    
    $np-q \leq moda \leq np+p$
    
    \textbf{В нашем случае:}
    
    $5 \cdot \frac{1}{6} - \frac{5}{6} \leq moda \leq 5 \cdot \frac{1}{6} + \frac{1}{6}$
    
    $0 \leq moda \leq 1$ то есть моды две - это 0 и 1
    
    \textbf{Ответ:} E.
    
    
    
    \item
    Несложно найти мат. ожидание для одного броска:
    
    $\mathbb{E}(X) = \frac{1}{6}\cdot(1 + 2 + 3 + 4 + 5 + 6) = \frac{21}{6}$
    
    Посчиатем мат. ожидание 5 бросоков:

    $Y = \sum X_i$ $\Rightarrow$ $\mathbb{E}(Y) = 5\mathbb{E}(X) = \frac{105}{6} = 17.5$
    
    \textbf{Ответ:} E.
    
    
    
    \item
    Мат. ожидания случайных величин $\xi, \eta $: $\mu_\xi = \mu_\eta = 0$
    
    На главной диагонали ковариационной матрицы расположены дисперсии случайных величин $\xi, \eta $: $D(\xi) = D(\eta) = 1$
    
    На побочной диагонали их ковариация: $Cov(\xi, \eta) = 0.5$
    
    Теперь можем найти $r = Corr(\xi, \eta) = \frac{Cov(\xi, \eta)}{\sqrt{Var(\xi)}\sqrt{Var(\eta)}} = \frac{0.5}{1} = 0.5$
    
    \textbf{Вспомним функцию плотности нормального двумерного распределения:}
    
    $f_{\xi, \eta}(x,y) = \frac{1}{2\pi\sqrt{D(\xi)}\sqrt{D(\eta)}\sqrt{1-r^2}}\cdot \exp{\{\frac{-1}{2(1-r^2)}(\frac{(x-\mu_\xi)^2}{D(\xi)}-\frac{2r(x-\mu_\xi)(y-\mu_\eta)}{\sqrt{D(\xi)}\sqrt{D(\eta)}}+\frac{(y-\mu_\eta)^2}{D(\eta)})\}}$
    
    \textbf{Подставим наши параметры:}
    
    $f_{\xi, \eta}(x,y) = \frac{1}{2\pi\sqrt{(1-0.5^2)}}\cdot \exp{\{\frac{-1}{1.5}(x^2-xy+y^2\}}$
    
    Откуда очевидно, что $a = \sqrt{(1-0.5^2)} = \sqrt{0.75} = \frac{\sqrt{3}}{2}$ из знаменателя, $b=1$, т.к. перед $xy$ в степени экспоненты не стоит множителя
    
    \textbf{Ответ:} C.
    
    
    
    
    
    \item
    \textbf{Уточнить}
    
    Заметим, что случайные величины $\xi, \eta $ стандартные нормальные, так как равны их параметры: $D(\xi) = D(\eta) = 1$ и $\mu_\xi = \mu_\eta = 0$
    
    \textbf{Попробуем решить методом исключения:}
    
    A) Хи-квадрат закон распределения случайной величины предполагает случайную величину, равную сумме стандартных нормальных случайных величин.
    
    $\eta$ - стандартная нормальная случайная велична, но $2\eta$ уже не будет являться стандартной, хотя и останется нормальной случайной величиной $\Rightarrow$ ответ A не подходит
    
    C) D)  ответы C и D эквивалентны, но здесь ответ единственный  $\Rightarrow$ ответы C и D не подходят
    
    E) $\xi$ - стандартная нормальная случайная велична, но если вычесть из нее некоторую другую случайную величину, то стандартной $\xi$ уже не будет $\Rightarrow$ ответ E не подходит
    
    Остался ответ B) - его и выбираем
    
    \textbf{Если решать не методом исключения}, то ответ B) также окажется верным:
    $z = (\xi-0.5\eta, \eta)^T$ второй элемент случайного вектора - стандартная нормальная случайная величина (известно из условия), первый элемент случайного вектора - также нормальная случайная величина, т.к. представляет собой линейную комбинацию нормальных случайных величин
    
    \textbf{Ответ:} B.
    
    
    
    \item
    \textbf{Уточнить}

    
    
    \item
    $\mathbb{E}(X \mid Y=0) = 0 \cdot P(X=0 \mid Y=0) + 2 \cdot P(X=2 \mid Y=0) = 0 \cdot \frac{1}{2} + 2 \cdot \frac{1}{2} = 1$
    
    \textbf{Ответ:} A.
    
    \item
    $P(X=0 \mid Y<1) = \frac{P(X=0 \cap Y<1)}{P(Y<1)} = \frac{\frac{1}{6}}{\frac{1}{3}+\frac{1}{6}+\frac{1}{6}} = 0.25$
    
    \textbf{Ответ:} B.
    
    
    \item
    $Var(Y) = E(Y^2) - E^2(Y)$
    
    $E(Y) = \frac{1}{3} \cdot (-1) + \frac{1}{3} \cdot 0 + \frac{1}{3} \cdot 1 = 0$ $\Rightarrow$ $E^2(Y)=0$
    
    $E(Y) = \frac{1}{3} \cdot (-1)^2 + \frac{1}{3} \cdot 0^2 + \frac{1}{3} \cdot 1^2 = 0$ $\Rightarrow$ $E^2(Y)= \frac{2}{3}$
    
    $Var(Y) = E(Y^2) - E^2(Y) = \frac{2}{3} - 0 = \frac{2}{3}$
    
    \textbf{Ответ:} A.
    
    \item
    $Cov(XY) = E(XY) - E(X)E(Y)$
    
    $E(XY) = 0\cdot(-1)\cdot0 + 2\cdot(-1)\cdot\frac{1}{3} + 0\cdot0\cdot\frac{1}{6} + 0\cdot0\cdot\frac{1}{6} + 0\cdot1\cdot\frac{1}{6} + 2\cdot1\cdot\frac{1}{6} = -\frac{1}{3}$
    
    $E(X) = 0\cdot\frac{1}{3} + 2\cdot\frac{2}{3} = \frac{4}{3}$
    
    $E(Y) =0 $ из предыдущего вопроса
    
    $Cov(XY) = -\frac{1}{3} - \frac{4}{3}\cdot0 = -\frac{1}{3}$
    
    \textbf{Ответ:} B.
    
    \item
    $P(X<0.5, Y<0.5) = \int_{0}^{0.5}\int_{0}^{0.5} 9x^2y^2 \,dx\,dy = \int_{0}^{0.5} \frac{3}{8}y^2 ,dy = \frac{1}{64}$
    
    \textbf{Ответ:} D.
    
    
    \item
    $f_{x\mid y=1} = \frac{f_{xy}}{f_{y}} $
    
    $f_{y} = \int_{0}^{1} 9x^2y^2 \,dx = 3y^2$
    
    $f_{x\mid y=1} = \frac{9x^2y^2}{3y^2} = 3x^2$
    
    \textbf{Ответ:} A.
    
    \item
    Пусть X=1 - событие в классе есть отличник
    
    Сначала выбирается класс (вероятность выбрать любой из классов равна $\frac{1}{3}$), затем в зависимости от класса вероятности будут отличаться, т.к. в них разная концентрация отличников
    
    $P(X=1) = 0.5\cdot\frac{1}{3} + 0.3\cdot\frac{1}{3} + 0.4\cdot\frac{1}{3} = 0.4$
    
    \textbf{Ответ:} E.
    
    \item
    По формуле условной вероятности:
    $P(A\mid B) = \frac{P(A\cap B)}{P(B)}$
    
    Или в другом виде:
    $P(A\cap B)=P(A\mid B)P(B) = 0.3 \cdot 0.5 = 0.15$
    
    $P(A\cup B) = P(A) + P(B) - P(A\cap B) = 0.2+0.5-0.15 = 0.55$
    
    \textbf{Ответ:} D.
    
    
    \item
    Вспомним формулу вероятности для биномиального распределения:
    
    $P(X>=2) = C_{10}^1 \cdot0.2^1 \cdot 0.8^9$
    
    \textbf{Ответ:} A.
    
    \item
    Пусть $W=1$ - событие "покупатель женщина", $M$ - случайная величина, показывающая сумму в чеке. Покупка была совершена на $1234$ рубля, то есть $M>1000$
    
    $P(W=1 \mid M>1000) = \frac{P(W=1 \cup M>1000)}{P(M>1000)}$
    
    Покупку совершила женщина ($W=1$) и сумма $M>1000$:
    
    $P(W=1 \cup M>1000) = 0.5 \cdot 0.6 = 0.3$

    Доли женщин и мужчин одинаковы, вычислим $P(M>1000)$:
    
    $P(M>1000) = 0.5 \cdot 0.6 + 0.5 \cdot 0.3 = 0.45$
    
    
    Следовательно:
    
    $P(W=1 \mid M>1000) = \frac{0.3}{0.45} = \frac{2}{3}$
    
    \textbf{Ответ:} C.
    
    
    \item
    Известно, что функция распределения:
    \begin{enumerate}
        \item существует для непрерывных случайных величин
        \item $F_X \in [0; 1]$
        \item при $x\rightarrow{-\infty}$ $F_X = 0$
        \item при $x\rightarrow{+\infty}$ $F_X = 1$
        \item $P(X \in (a; b]) = F_X(b) - F_X(a)$
    \end{enumerate} 
    
    \textbf{Ответ:} D.
    
    
\end{enumerate}



\end{document}
