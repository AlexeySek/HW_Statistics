%I used the template from Coursera course on LaTex by Danil Fyodorovykh%

\documentclass[a4paper]{article} % this is used for comments
\usepackage[utf8]{inputenc}
%%% Дополнительная работа с математикой
\usepackage{amsmath,amsfonts,amssymb,amsthm,mathtools} % AMS
\usepackage{icomma} % "Умная" запятая: $0,2$ --- число, $0, 2$ --- перечисление
\usepackage[english,russian]{babel}

%% Номера формул
\mathtoolsset{showonlyrefs=true} % Показывать номера только у тех формул, на которые есть \eqref{} в тексте.

%% Шрифты
\usepackage{euscript}	 % Шрифт Евклид
\usepackage{mathrsfs} % Красивый матшрифт

%% Свои команды
\DeclareMathOperator{\sgn}{\mathop{sgn}}

%% Перенос знаков в формулах (по Львовскому)
\newcommand*{\hm}[1]{#1\nobreak\discretionary{}
{\hbox{$\mathsurround=0pt #1$}}{}}

\DeclareMathOperator{\Lin}{\mathrm{Lin}}
\DeclareMathOperator{\Linp}{\Lin^{\perp}}
\DeclareMathOperator*\plim{plim}
\DeclareMathOperator{\grad}{grad}
\DeclareMathOperator{\card}{card}
\DeclareMathOperator{\sgn}{sign}
\DeclareMathOperator{\sign}{sign}

\DeclareMathOperator*{\argmin}{arg\,min}
\DeclareMathOperator*{\argmax}{arg\,max}
\DeclareMathOperator*{\amn}{arg\,min}
\DeclareMathOperator*{\amx}{arg\,max}
\DeclareMathOperator{\cov}{Cov}
\DeclareMathOperator{\Var}{Var}
\DeclareMathOperator{\Cov}{Cov}
\DeclareMathOperator{\Corr}{Corr}
\DeclareMathOperator{\pCorr}{pCorr}
\DeclareMathOperator{\E}{\mathbb{E}}
\let\P\relax
\DeclareMathOperator{\P}{\mathbb{P}}


\newcommand{\cN}{\mathcal{N}}
\newcommand{\cU}{\mathcal{U}}
\newcommand{\cBinom}{\mathcal{Binom}}
\newcommand{\cBin}{\cBinom}
\newcommand{\cPois}{\mathcal{Pois}}
\newcommand{\cBeta}{\mathcal{Beta}}
\newcommand{\cGamma}{\mathcal{Gamma}}

\newcommand \R{\mathbb{R}}
\newcommand \N{\mathbb{N}}
\newcommand \Z{\mathbb{Z}}





\newcommand{\dx}[1]{\,\mathrm{d}#1} % для интеграла: маленький отступ и прямая d
\newcommand{\ind}[1]{\mathbbm{1}_{\{#1\}}} % Индикатор события
%\renewcommand{\to}{\rightarrow}
\newcommand{\eqdef}{\mathrel{\stackrel{\rm def}=}}
\newcommand{\iid}{\mathrel{\stackrel{\rm i.\,i.\,d.}\sim}}
\newcommand{\const}{\mathrm{const}}


% вместо горизонтальной делаем косую черточку в нестрогих неравенствах
\renewcommand{\le}{\leqslant}
\renewcommand{\ge}{\geqslant}
\renewcommand{\leq}{\leqslant}
\renewcommand{\geq}{\geqslant}







\title{Промежуточный экзамен 2016-2017}
\author{Алексей Сек, БЭК182}
\date{\today}






\begin{document}

\maketitle

\textbf{Промежуточный экзамен 2016-2017}

\begin{enumerate}

    %Задача 1
    \item
    Известно, что некоторые события $A$ и $B$ независимы, если выполняется такое условие:
    $ \P(A \cap B) = \P(A) \cdot \P(B) $, верно и обратное: если данное условие не выполняется — события являются зависимыми. Исходя из этого, решим задачу:
    
    \textbf{Посчитаем вероятности каждого из событий отдельно:}
    
    Вполне очевидно, что троек в колоде ровно 4, тогда по классическому определению вероятности:
    \[ 
    \P(A) = \frac{4}{52} = \frac{1}{13} 
    \]
    
    Семерок в колоде, что также очевидно, ровно 4, но т.к. событие заключается в вытаскивании второй карты, то вероятность будет иной.
    Примем во внимание, что первой картой могла быть как семерка, так и не семерка, вычислим вероятность события $B$: 
    \[
    \P(B) = \frac{4}{52}\cdot\frac{3}{51} + \frac{48}{52}\cdot\frac{4}{51} 
    \]
    
    Если третья карта — дама пик, то первые две карты — не дамы пик, а дама пик в колоде всего одна, следовательно:
    \[ 
    \P(C) = \frac{51}{52}\cdot\frac{50}{51}\cdot\frac{1}{50} 
    \]
    
    \textbf{Посчитаем все произведения пар событий (от перестановок множителей сумма не меняется — поэтому считаем только 3 пары):}
    
    \[
    \P(A) \cdot \P(B) = \frac{4}{52}\cdot(\frac{4}{52}\cdot\frac{3}{51} + \frac{48}{52}\cdot\frac{4}{51}) 
    \]
    
    \[
     \P(A) \cdot \P(C) = \frac{4}{52}\cdot\frac{51}{52}\cdot\frac{50}{51}\cdot\frac{1}{50} 
    \]
    
    \[
     \P(B) \cdot \P(C) = (\frac{4}{52}\cdot\frac{3}{51} + \frac{48}{52}\cdot\frac{4}{51})\cdot\frac{51}{52}\cdot\frac{50}{51}\cdot\frac{1}{50}
    \]
    
    \textbf{Теперь посчитаем пересечения рассмотренных событий:}
    
    \[
    \P(A \cap B) = \frac{4}{52}\cdot\frac{4}{51}
    \]
    
    \[
    \P(A \cap C) = \frac{4}{52}\cdot\frac{50}{51}\cdot\frac{1}{50} 
    \]
    
    Во второй части мы считаем, что первая карта не семерка и не дама пик, таких карт $52-4-1=47$:
    \[
    \P(B \cap C) = \frac{4}{52}\cdot\frac{3}{51}\cdot\frac{1}{50} + \frac{47}{52}\cdot\frac{4}{51}\cdot\frac{1}{50}
    \] 
    
    \textbf{Сравним вероятности пересечений событий и произведения вероятностей этих событий:}
    
    $ \P(A) \cdot \P(B) \neq \P(A \cap B) $ $\Rightarrow$ $A$ и $B$ — зависимые события
    
    $ \P(A) \cdot \P(C) \neq \P(A \cap C) $ $\Rightarrow$ $A$ и $C$ — зависимые события
    
    $ \P(B) \cdot \P(C) \neq \P(B \cap C) $ $\Rightarrow$ $B$ и $C$ — зависимые события
    
    \textbf{Ответ:} B.
    
    
    %Задача 2
    \item
    Известно, что функция плотности $f(x)$ обладает следующими свойствами:
    \begin{itemize}
        \item $f(x) \ge 0$, для $ \forall x$ — вероятность не может быть отрицательной
        \item $f(x)$ — непрерывна в области опрелеления
        \item $ \int_{-\infty}^{+\infty} f(x) dx = 1$ — условие нормировки (вероятность от 0 до 1)
    \end{itemize}
    
    \textbf{Рассмотрим каждую из функций на выполнение указанных свойств:}
    
    A. $f(x) = -1 \le 0$ $\Rightarrow$ не подходит
    
    B. $f(x) \le 0$ например, при $x = 0$ $\Rightarrow$ не подходит
    
    C. $ \int_{-\infty}^{+\infty} f(x) dx = \int_{-1}^{+\infty} dx/x^2 = 1 $ $\Rightarrow$ все свойтсва соблюдаются
    
    D. Данная функция похожа на функцию плотности для нормального распределения, но из-за отсутствия делителя в степени экспоненты - интеграл нельзя посчитать $\Rightarrow$ не подходит
    
    E. $ \int_{-\infty}^{+\infty} f(x) dx = \int_{0}^{2} x^2 dx = \frac{8}{3} \neq 1 $ $\Rightarrow$ не подходит
    
    \textbf{Ответ:} C.
    
    
    %Задача 3
    \item
    Известно, что:
    \[
    \E(XY) = \Cov(X,Y) + \E(X)\E(Y)
    \]
    
    Следовательно:
    \[
    \E(XY) = 2 + 3\cdot2 = 8
    \]
    
    \textbf{Ответ:} A.
    
    
    %Задача 4
    \item 
    Известно, что:
    \[
    \Corr(X,Y) = \frac{\Cov(X,Y)}{\sqrt{\Var(X)}\sqrt{\Var(Y}}
    \]
    
    Следовательно:
    \[
    \Corr(X,Y) = \frac{2}{\sqrt{12}\sqrt{1}} = \frac{1}{\sqrt{3}}
    \]
    
    \textbf{Ответ:} A.
    
    
    %Задача 5
    \item
    Изввестно, что, если $a, b, c$ — некоторые константы, $X, Y$ — некоторые случайные величины:
    \[
    \Var(aX + bY + c) = a^2 \Var(X) + b^2 \Var(Y) + 2ab\Cov(X,Y)
    \]
    
    Следовательно:
    \[
    \Var(2X - Y + 4) = 2^2 \cdot 12 + 1 \cdot 1 - 4 \cdot 2 = 48 + 1 - 8 = 41
    \]
    
    \textbf{Ответ:} E.
    
    
    %Задача 6
    \item
    Известно, что в ковариационной матрице на главной диагонали стоят дисперсии случайных величин, а на побочной — ковариации данных случайных величин друг с другом
    
    Если матрица единичная, то на главной диагонали стоят единицы: $\Var(X) = 1$ и $\Var(Y) = 1$, а на побочной - нули:  $\Cov(X,Y) = 0$
    
    Если ковариация равна нулю, то случайные величины независимы, что и утверждается в варианте D
    
    \textbf{Ответ:} D.
    

    %Задача 7
    \item
    Для решения вспомним свойства корреляции и ковариации:
    
    \[
    \begin{gathered} Corr(X+Y, 2Y-7) = \frac{\Cov(X+Y, 2Y-7)}{\sqrt{\Var(X+Y)}\sqrt{\Var(2Y-7)}} = \\ 
    =\frac{\frac{\Cov(X, 2Y) + 2 \Var(Y)}{\sqrt{\Var(X)}\sqrt{\Var(Y)}}}{\frac{\sqrt{\Var(X) + \Var(Y) + 2\Cov(X,Y)}\sqrt{4\Var(Y)}}{\sqrt{\Var(X)}\sqrt{\Var(Y)}}} = \\ 
    =\frac{\frac{2\Cov(X, Y)}{\Var(X) + \Var(Y)} + \frac{2 \Var(Y)}{\Var(X) + \Var(Y)}}{2\sqrt{1 + \frac{\Var(Y)}{\Var(X)} + \frac{2\Cov(X,Y)}{\Var(X)}}} = \\ 
    =\frac{2 \cdot 0.5 + 2 \cdot \sqrt{\frac{\Var(Y)}{\Var(Y)}}}{2\sqrt{1 + 1 + 1}} = \frac{1 + 2}{2 \sqrt{3}} = \frac{\sqrt{3}}{2} \end{gathered} 
    \]
    
    
    Промежуточные подсчеты: 
    \[
    \Corr(X,Y) = 0.5; \Var(X) = \Var(Y) \Rightarrow \Cov(X,Y) = \frac{\sqrt{\Var(X)}\sqrt{\Var(Y)}}{2}
    \]
    
    \textbf{Ответ:} B.
    
    
    %Задача 8
    \item
    И так несложная задача сильно упроститься, если представить отрезок от 0 до 1, на котором случайная величина $\xi$ может равномерно распределена. 
    
    С какой же вероятностью она попадет в часть этого отрезка, ограниченную точками $0.2$ и $0.7$?
    
    Очевидно, что длина такой части отрезка равна $0.5$, когда длина всего отрезка от $0$ и $1$ равна 1
    
    Очевидно: 
    \[
    \P(0.2 < \xi < 0.7) = \frac{0.7 - 0.2}{1 - 0} = 0.5
    \]
    
    \textbf{Ответ:} D.
    
    
    %Задача 9
    \item
    По ЦПТ указанное распределение сходится к стандартному нормальному распределению $\cN(0,1)$ при $n \rightarrow \infty$
    
    Нам известна функция плотности стандартной нормальной случайной величины, а вероятность, как известно, находится как определенный интеграл от функции плотности. 
    
    Пределы интегрирования равны: $1$ и ${+\infty}$, т.к. исходя из условия нам интересны значения $x > 1$
    
    Очевидно, что подойти может только ответ B, который представляет интеграл с правильными пределами интегрирования от правильной функцмм плотности
    
    \textbf{Ответ:} B.
    
    
    %Задача 10
    \item
    \textbf{Известно из условия, что $n = 100$:}
    
    Используем свойство математического ожидания суммы:
    \[
    \E(S_n) = \E(X_i) \cdot n = 400 \cdot 100 = 40000
    \]
    
    Так как случайные величины независимые:
    \[
    \Var(S_n) = \Var(\sum X_i) = n * \Var(X_i) = 40000
    \]
    
    
    
    Применим ЦПТ:
    
    \[
    \begin{gathered} \P(S_n > 40400) = \P(\frac{S_n - \E(S_n)}{\sqrt{\Var(S_n)}} > \frac{40400 - 40000}{\sqrt{40000}}) = \P(\cN(0,1) > 2)= \\
    = 1 -\P(\cN(0,1) < 2) = 1 - 0.9772 = 0.0228 \end{gathered}
    \]
    
    В ответах указать приближенный ответ, там есть 0.0227, что нам подходит
    
    \textbf{Ответ:} D.
    
    
    %Задача 11
    \item
    По условию дано: $\E(X) = 10000$
    
    Неравенство Макркова:
    
    \[
    \P(\mid X \mid > a) \leq \frac{\E(X)}{a}
    \]
    
    В нашем случае:
    
    \[
    \P(\mid X \mid > 50000) \leq \frac{10000}{50000} = 0.2
    \]
    
    \textbf{Ответ:} A.
    
    
    %Задача 12
    \item
    Несовместность событий $\Leftrightarrow$ они не могут произойти вместе
    
    \textbf{Всего 3 подбрасывания монеты:}
    
    Событие A: хотя бы 1 раз решка

    Событие B: хотя бы 1 раз орел

    Событие C: все три раза орел
    
    \textbf{Применим здравый смысл:}
    
    A и B одновременно произойти могут: например, 1 орел, 2 решки $\Rightarrow$ Совместны
    
    A и C одновременно произойти не могут: если 3 орла из 3, то решек 0 $\Rightarrow$ Несовместны
    
    B и C одноврменно произойти могут: если 3 раза орел, то хотя бы 1 орел точно есть $\Rightarrow$ Совместны
    
    \textbf{Ответ:} A.
    
    
    %Задача 13
    \item
    По условию математическое ожидание:
    \[
    \E(X) = 5\cdot10^4
    \]
    
    Дано стандартное отклонение, а нужна дисперсия:
    \[
    \Var(X) = \sigma^2 = (10^4)^2 = 10^8
    \]
    
    Неравенство Чебышёва:
    \[
    \P(\mid X - \E(X) \mid > a) \leq \frac{\Var(X)}{a^2}
    \]
    
    В нашем случае:
    \[
    \begin{gathered} \P(\mid X - 5\cdot10^4 \mid \leq 2\cdot10^4) = 1 -  \P(\mid X - 5\cdot10^4 \mid > 2\cdot10^4) = \\
    =1 - \frac{10^8}{(2* 10^4)^2} = 1 - 0.25 = 0.75 \end{gathered}
    \]
    
    \textbf{Ответ:} C.
    
    
    %Задача 14
    \item
    Мат. ожидание биномиального распределения:
    \[
    \E((\xi_1)^{2016}) = 0.6
    \]
    
    По Закону Больших Чисел:
    
    \[
    \plim_{n\to\infty} \frac{(\xi_1)^{2016} + ... + (\xi_n)^{2016}}{n} = \E((\xi_1)^{2016}) = 0.6
    \]
    
    \textbf{Ответ:} A.
    
    
    %Задача 15
    \item
    Ровно $2$ раза шестерка $\Rightarrow$ $2$ раза шестерка и  $3$ раза не шестерка
     
    События независимые $\Rightarrow$ можем взять их произведение
    
    Пусть $X$ количество выпавших шестерок, тогда из функции вероятности биномиального распределения:
    \[
    \P(X=2) = C_5^2 \cdot(\frac{1}{6})^2 \cdot (\frac{5}{6})^3 = \frac{5^4}{2^4\cdot3^5}
    \]
    
    \textbf{Ответ:} Нет верного ответа.
    
    
    %Задача 16
    \item
    Несложно найти мат. ожидание для одного броска:
    \[
    \E(X) = \frac{1}{6}\cdot1 + \frac{5}{6}\cdot0 = \frac{1}{6}
    \]
    
    Посчиатем мат. ожидание 5 бросоков:
    \[
    Y = \sum_{i=1}^{5} X_i \Rightarrow \E(Y) = 5\E(X) = \frac{5}{6}
    \]
    
    Биномиальное распределение, поэтому:
    \[
    \Var(Y) = np(1-p) = 5 \cdot \frac{1}{6} \cdot \frac{5}{6} = \frac{25}{36}
    \]
    
    \textbf{Ответ:} Нет верного ответа.
    
    
    %Задача 17
    \item
    Биномиальное распределение, поэтому:
    \[
    np-q \leq moda \leq np+p
    \]
    
    В нашем случае:
    \[
    5 \cdot \frac{1}{6} — \frac{5}{6} \leq moda \leq 5 \cdot \frac{1}{6} + \frac{1}{6}
    \]
    
    То есть моды две — это 0 и 1:
    \[
    0 \leq moda \leq 1
    \]
    
    \textbf{Ответ:} E.
    
    
    %Задача 18
    \item
    Несложно найти мат. ожидание для одного броска:
    \[
    \E(X) = \frac{1}{6}\cdot(1 + 2 + 3 + 4 + 5 + 6) = \frac{21}{6}
    \]
    
    Посчиатем мат. ожидание 5 бросоков:
    \[
    Y = \sum_{i=1}^5 X_i \Rightarrow \E(Y) = 5\E(X) = \frac{105}{6} = 17.5
    \]
    
    \textbf{Ответ:} E.
    
    
    %Задача 19
    \item
    Мат. ожидания случайных величин $\xi, \eta $ даны в условии :
    \[
    \E(\xi) = \E(\eta) = 0
    \]
    
    На главной диагонали ковариационной матрицы расположены дисперсии случайных величин $\xi, \eta $: 
    \[
    \Var(\xi) = \Var(\eta) = 1
    \]
    
    На побочной диагонали их ковариация: 
    \[
    \Cov(\xi, \eta) = 0.5
    \]
    
    Теперь можем найти корреляцию:
    \[
    r = \Corr(\xi, \eta) = \frac{\Cov(\xi, \eta)}{\sqrt{\Var(\xi)}\sqrt{\Var(\eta)}} = \frac{0.5}{1} = 0.5
    \]
    
    Вспомним функцию плотности нормального двумерного распределения:
    
    \[
    \begin{gathered} f_{\xi, \eta}(x,y) = \frac{1}{2\pi\sqrt{\Var(\xi)\Var(\eta)(1-r^2)}} \times \\ 
    \times \exp{\{\frac{-1}{2(1-r^2)}(\frac{(x-\E(\xi))^2}{\Var(\xi)}-\frac{2r(x-\E(\xi))(y-\E(\eta))}{\sqrt{\Var(\xi)\Var(\eta)}}+\frac{(y-\E(\eta))^2}{\Var(\eta)})\}} \end{gathered}
    \]
    
    Подставим наши параметры:
    \[
    f_{\xi, \eta}(x,y) = \frac{1}{2\pi\sqrt{(1-0.5^2)}} \exp{\{\frac{-1}{1.5}(x^2-xy+y^2\}}
    \]
    
    Откуда $a$ очевидно из знаменателя первого множителя, $b$ очевидно, т.к. в степени экспоненты не стоит множителя:
    \[
    a = \sqrt{(1-0.5^2)} = \sqrt{0.75} = \frac{\sqrt{3}}{2}, b=1
    \] 
    
    \textbf{Ответ:} C.
    
    
    %Задача 20
    \item
    \textbf{Уточнить}
    
    Заметим, что случайные величины $\xi, \eta $ стандартные нормальные, так как их параметры: $\Var(\xi) = \Var(\eta) = 1$ и $\E(\xi) = \E(\eta) = 0$
    
    \textbf{Попробуем решить методом исключения:}
    
    A) Хи-квадрат закон распределения случайной величины предполагает случайную величину, равную сумме стандартных нормальных случайных величин.
    
    $\eta$ - стандартная нормальная случайная велична, но $2\eta$ уже не будет являться стандартной, хотя и останется нормальной случайной величиной $\Rightarrow$ ответ A не подходит
    
    C) D)  ответы C и D эквивалентны, но здесь ответ единственный  $\Rightarrow$ ответы C и D не подходят
    
    E) $\xi$ - стандартная нормальная случайная велична, но если вычесть из нее некоторую другую случайную величину, то стандартной $\xi$ уже не будет $\Rightarrow$ ответ E не подходит
    
    Остался ответ B) - его и выбираем
    
    \textbf{Если решать не методом исключения}, то ответ B) также окажется верным:
    $z = (\xi-0.5\eta, \eta)^T$ второй элемент случайного вектора — стандартная нормальная случайная величина (известно из условия), первый элемент случайного вектора — также нормальная случайная величина, т.к. представляет собой линейную комбинацию нормальных случайных величин
    
    \textbf{Ответ:} B.
    
    
    %Задача 21
    \item
    \textbf{Уточнить}

    
    %Задача 22
    \item
    Найдем условное математическое ожидание:
    \[
    \E(X \mid Y=0) = 0 \cdot \P(X=0 \mid Y=0) + 2 \cdot \P(X=2 \mid Y=0) = 0 \cdot \frac{1}{2} + 2 \cdot \frac{1}{2} = 1
    \]
    
    \textbf{Ответ:} A.
    
    
    %Задача 23
    \item
    Найдем условную вероятность:
    \[
    \P(X=0 \mid Y<1) = \frac{\P(X=0 \cap Y<1)}{\P(Y<1)} = \frac{\frac{1}{6}}{\frac{1}{3}+\frac{1}{6}+\frac{1}{6}} = 0.25
    \]
    
    \textbf{Ответ:} B.
    
    
    %Задача 24
    \item
    Известно, что дисперсию можно найти по формуле:
    \[
    \Var(Y) = \E(Y^2) - \E^2(Y)
    \]
    
    Найдем квадрат мат. ожидания случайной величины:
    \[
    \E(Y) = \frac{1}{3} \cdot (-1) + \frac{1}{3} \cdot 0 + \frac{1}{3} \cdot 1 = 0 \Rightarrow E^2(Y)=0
    \]
    
    Найдем мат. ожидание квадрата случайной величины:
    \[
    \E(Y^2) = \frac{1}{3} \cdot (-1)^2 + \frac{1}{3} \cdot 0^2 + \frac{1}{3} \cdot 1^2 = 0 \Rightarrow \E^2(Y)= \frac{2}{3}
    \]
    
    Найдем искомую дисперсию:
    \[
    \Var(Y) = \E(Y^2) - \E^2(Y) = \frac{2}{3} - 0 = \frac{2}{3}
    \]
    
    \textbf{Ответ:} A.
    
    
    %Задача 25
    \item
    Известно, что ковариация может быть найдена по следующей формуле:
    \[
    \Cov(XY) = \E(XY) - \E(X)\E(Y)
    \]
    
    Найдем мат. ожидание произведения случайных величин $X$ и $Y$:
    \[
    \E(XY) = 0\cdot(-1)\cdot0 + 2\cdot(-1)\cdot\frac{1}{3} + 0\cdot0\cdot\frac{1}{6} + 0\cdot0\cdot\frac{1}{6} + 0\cdot1\cdot\frac{1}{6} + 2\cdot1\cdot\frac{1}{6} = -\frac{1}{3}
    \]
    
    Найдем мат. ожидание случайной величины $X$:
    \[
    \E(X) = 0\cdot\frac{1}{3} + 2\cdot\frac{2}{3} = \frac{4}{3}
    \]
    
    Найдем мат. ожидание случайной величины $Y$:
    \[
    \E(Y) = \frac{1}{3} \cdot (-1) + \frac{1}{3} \cdot 0 + \frac{1}{3} \cdot 1 = 0
    \]
    
    Найдем искомую ковариацию:
    \[
    \Cov(XY) = -\frac{1}{3} - \frac{4}{3}\cdot0 = -\frac{1}{3}
    \]
    
    \textbf{Ответ:} B.
    
    
    %Задача 26
    \item
    Известно, что вероятность может быть найдена как интеграл по заданной области:
    \[
    \P(X<0.5, Y<0.5) = \int_{0}^{0.5}\int_{0}^{0.5} 9x^2y^2 \,dx\,dy = \int_{0}^{0.5} \frac{3}{8}y^2 ,dy = \frac{1}{64}
    \]
    
    \textbf{Ответ:} D.
    
    
    %Задача 27
    \item
    Известно, что условная функция плотности может быть найдена по следующей формуле:
    \[
    f_{x\mid y=1} = \frac{f_{xy}}{f_{y}}
    \]
    
    Найдем функцию плотности случайной величины $Y$ из совместной функции плотности:
    \[
    f_{y} = \int_{0}^{1} 9x^2y^2 \,dx = 3y^2
    \]
    
    Найдем искомую условную функцию плотности:
    \[
    f_{x\mid y=1} = \frac{9x^2y^2}{3y^2} = 3x^2
    \]
    
    \textbf{Ответ:} A.
    
    
    %Задача 28
    \item
    Пусть $X=1$ — событие в классе есть отличник
    
    Сначала выбирается класс (вероятность выбрать любой из классов равна $\frac{1}{3}$), затем в зависимости от класса вероятности будут отличаться, т.к. в них разная концентрация отличников:
    
    Искомая вероятность может быть вычислена таким образом:
    \[
    \P(X=1) = 0.5\cdot\frac{1}{3} + 0.3\cdot\frac{1}{3} + 0.4\cdot\frac{1}{3} = 0.4
    \]
    
    \textbf{Ответ:} E.
    
    
    %Задача 29
    \item
    По формуле условной вероятности:
    \[
    \P(A\mid B) = \frac{\P(A\cap B)}{\P(B)}
    \]
    
    Или в другом виде:
    \[
    \P(A\cap B)=\P(A\mid B)\P(B) = 0.3 \cdot 0.5 = 0.15
    \]
    
    Вероятность объединения событий:
    \[
    \P(A\cup B) = \P(A) + \P(B) - \P(A\cap B) = 0.2+0.5-0.15 = 0.55
    \]
    
    \textbf{Ответ:} D.
    
    
    %Задача 30
    \item
    Из формулы вероятности для биномиального распределения:
    \[
    \P(X>=2) = C_{10}^1 \cdot0.2^1 \cdot 0.8^9
    \]
    
    \textbf{Ответ:} A.
    
    
    %Задача 31
    \item
    Пусть $W=1$ — событие "покупатель женщина", $M$ — случайная величина, показывающая сумму в чеке. 
    Покупка была совершена на $1234$ рубля, то есть $M>1000$
    
    Вероятность покупки женищной при условии $M>1000$:
    \[
    \P(W=1 \mid M>1000) = \frac{\P(W=1 \cup M>1000)}{\P(M>1000)}
    \]
    
    Вероятность объединения событий ($W=1$) и $M>1000$:
    \[
    \P(W=1 \cup M>1000) = 0.5 \cdot 0.6 = 0.3
    \]

    Доли женщин и мужчин одинаковы, вычислим $\P(M>1000)$:
    \[
    \P(M>1000) = 0.5 \cdot 0.6 + 0.5 \cdot 0.3 = 0.45
    \]
    
    Следовательно, искомая вероятность равна:
    \[
    \P(W=1 \mid M>1000) = \frac{0.3}{0.45} = \frac{2}{3}
    \]
    
    \textbf{Ответ:} C.
    
    
    %Задача 32
    \item
    Известно, что функция распределения обладает следующими свойствами:
    \begin{enumerate}
        \item существует для непрерывных случайных величин
        \item $F_X \in [0; 1]$
        \item при $x\rightarrow{-\infty}$ $F_X = 0$
        \item при $x\rightarrow{+\infty}$ $F_X = 1$
        \item $\P(X \in (a; b]) = F_X(b) - F_X(a)$
    \end{enumerate} 
    
    Проверим каждый из вариантов на соответствие свойствам функции распределения: подходит только вариант D.
    
    \textbf{Ответ:} D.
    
    
\end{enumerate}



\end{document}
